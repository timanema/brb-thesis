\section{Introduction}
% \begin{itemize}
% \item Introduce the topic and explain why it is important (motivation!).

% \item Relate to the most relevant existing work from the literature (use BibTeX) \cite{example}, explain their contributions, and (critically) indicate what is still unanswered. 


% \item Explain what the research questions for this work are. 
% This usually is a subset of the unanswered questions. 

% \item Summarize the main contributions/conclusions of this research.
% NB: Make sure the title of the paper is a good match to the main research question / contribution / conclusion.

% \item Briefly indicate how the rest of the paper fits together to answer the research question.
% \end{itemize}

% For a longer research paper, a section with more elaborate discussion of the literature may follow, but for short (conference) submissions, this is often included in the introduction.


Distributed systems are at the heart of our everyday lives. These systems consist of autonomous processes that communicate with a subset of other processes in order to coordinate their efforts. This means that these systems have to be robust against arbitrary behaviour that some faulty or malicious nodes might exhibit. Fault-tolerant distributed communication algorithms are being used in practice to give this guarantee. 

The Byzantine fault model is often used to describe these fault-tolerant systems. In this model there are two types of processes: correct processes which follow their programming faithfully, and Byzantine processes which exhibit arbitrary behaviour which can consist of altering messages, creating new ones, or dropping messages all together.

There are several solutions to this problem, all of which make different assumptions and differ in their guarantees. An example of this is Dolev's \textit{reliable communication} (RC) algorithm~\cite{dolev}, which assumes a $2f+1$-connected network. Another example is Bracha's double echo authenticated broadcast~\cite{bracha}, which assumes a fully-connected network. The state-of-the-art solution for \textit{Byzantine Reliable Broadcasts} (BRB) described in \cite{bracha-dolev} and improved in \cite{bonomi2021practical} relies on an optimized combination of Dolev's RC algorithm \cite{bonomi2019multihop} and Bracha's double echo authenticated broadcast.

This research will focus on optimizing Dolev, Bracha, and Bracha-Dolev by minimizing the amount of redundant messages transmitted when the topology of the network is known to all processes. While the problem of reducing the amount of messages has been discussed in several papers, they focus on unknown network topologies~\cite{dolev-improvement,bonomi2019multihop,bonomi2021practical}, introduce cryptography and/or \textit{public-key infrastructure} (PKI)~\cite{signatures-crypo-1,pki-crypto-2}, or use trusted nodes~\cite{using-tee}. Focusing on the case where the network topology is known to all processes is worth investigating, as this is a realistic use-case and might allow for more optimizations. In addition to this, other papers have shown ways to reconstruct the topology~\cite{topology-discovery}, which makes it possible to use the optimizations in this paper for previously unknown topologies.
Even though the aforementioned papers do not assume a known network, most of their optimizations also apply.

In this paper we will start from solutions using naive routing and make the following contributions:\\
(i) We explain how a routing table can be created for Dolev using a combination of existing algorithms.\\
(ii) We discuss how the verification step of Dolev is trivial in our system.\\
(iii) We introduce XXX novel modifications to Dolev, Bracha, and Bracha-Dolev.\\
(iv) We present a detailed performance analysis using our profiling tool.

The structure of this paper is as follows. We will first explain what work has already been done in this field. Section~\ref{system-model} will introduce the system model and the problem, while also providing some background on Dolev, Bracha and Bracha-Dolev. Sections~\ref{contr-dolev},\ref{contr-bracha} and \ref{contr-bracha-dolev} will then introduce our novel modifications for Dolev, Bracha and Bracha-Dolev respectively. Section~\ref{eval} contains our performance analysis. We will briefly discuss the broader impact and reproducibility of our work in Section~\ref{broader-impact}. Finally, Section~\ref{conclusion} will conclude our paper.

\subsection*{Related work}
The idea of reliably reaching an agreement in the presence of faulty or malicious processes was first introduced by Lampert et al.~\cite{lamport2019byzantine}, and was named the \textit{Byzantine Agreement}. The network tolerance to faults can represented as $f$, which represents the amount of Byzantine processes that can be present before correct processes can no longer reliably communicate with each other. One can imagine this number heavily depends on the connectivity, i.e. the amount of nodes that can fail before the network is partitioned, of the network. A simple connected (1-connected) network will already be partitioned when a single Byzantine process exists, while a fully connected network (n-connected) can sustain more Byzantine nodes. Pease et al. proved that there exists a tight upperbound for $f$ in these networks, namely $f < \floor{N/3}$~\cite{pease1980reaching}.

When a network is partially connected, Dolev showed that processes can still communicate in the presence of $f$ Byzantine nodes when the network is at least $2f+1$-connected~\cite{dolev}. In this solution a message is flooded over the network, therefore following at least $2f+1$ vertex-disjoint paths. Since authenticated links\footnote{Authenticated links guarantee messages sent over a link originate from the complementing process} are assumed in this solution, every process can append the transmitter of a message to a header representing the message path. A process delivers a message when it has received the same payload data over $f+1$ vertex-disjoint paths. Note that this means it is possible for a Byzantine sender to cause a single correct process to deliver a message, violating the basic principles of a broadcast.

Bracha described the \textit{authenticated double echo} protocol~\cite{bracha} for fully connected networks, which gives the additional guarantee that either every correct process will deliver a message or none will. This protocol uses three phases -namely \textit{send}, \textit{echo}, and \textit{ready}- to coordinate the global acceptance of a message $m$.

In their original versions, both protocols are less than practical. In the case of Dolev, the worst-case message and computational complexity is high, making it impractical for large ($n=100$) networks. While Bracha is computationally less expensive, it requires a fully connected network, reducing its applicability in regular networks.

Bonomi et al.~\cite{bonomi2019multihop} introduced several improvements to Dolev's original protocol, considerably improving its average message complexity. These modifications make Dolev more practical for use in general networks, even though the complexity of delivery verification is still high. 
% The following modifcations were introduced:
% \begin{itemize}
%     \item If process $p$ receives a message $m$ directly from the source $s$ over an authenticated link, then $p$ will directly deliver the message.
%     \item If a message $m$ has been delivered by a process $p$, then it can discard all related Dolev paths and instead use an empty path when relaying.
%     \item Process $p$ only relays messages to neighbours that have not yet delivered it.
%     \item If process $p$ receives an empty path from a neighbour $q$, then it no longer has to relay and analyse messages to and from $q$.
%     \item Process $p$ stops relaying messages after its contents have been delivered and the empty path has been forwarded.
% \end{itemize}

Wang and Wattenhofer~\cite{bracha-dolev} introduced a new protocol, which combines Bracha and Dolev in order to use a protocol designed for a fully connected network (e.g. Bracha's protocol) on a k-connected (where $k < |V|$) network. More recently, Bonomi et al.~\cite{bonomi2021practical} introduced several novel improvements to this protocol and combined it with an optimized version of Dolev's RC protocol~\cite{dolev-improvement,bonomi2019multihop}. Their experiments showed promising results, and the possibility of extending these to other protocols exists. 
% Let us recall their modifications with their original identification:
% \begin{itemize}
%     \item \textbf{MBD.1}: Limit the payload data transmission by associating local IDs to payload data. When process $p$ sends payload data it includes a generated local ID, and only uses that local ID for further transmissions of the same payload data. 
%     \item \textbf{MBD.2}: When process $p$ receives Bracha's \textit{send} message it will not propogate it, but instead switch to regular \textit{echo} messages. Other processes will implicitly receive a \textit{send} message when they receive an \textit{echo} message.
%     \item \textbf{MBD.3}: When process $p$ needs to transmit two echo messages with an empty path to the same neighbour, the messages are merged into a single \textit{echo\_echo} message. This situation can occur when a process transmits an \textit{echo} message after a Dolev-deliver, and another \textit{echo} message because of said Dolev-deliver with echo amplification\footnote{In addition to ready amplification~\citationneeded, echo amplification can also applied}.
%     \item \textbf{MBD.4}: Similar to \textit{echo\_echo} messages, a process can also transmit a \textit{echo\_ready} message when a Dolev-deliver causes an additional echo, and a transition to the \textit{ready} state.
%     \item \textbf{MBD.6}: When process $p$ Dolev-delivers a \textit{ready} message originating from process $q$, echo messages originating from $q$ can be ignored by $p$.
%     \item \textbf{MBD.7}: When process $p$ Bracha-delivers a message $m$, it can ignore and discard all \textit{echo} messages related to $m$.
%     \item \textbf{MBD.8}: When process $p$ has Dolev-delivered a \textit{ready} message from neighbouring process $q$, it can abstain from sending \textit{echo} messages to $q$.
%     \item \textbf{MBD.9}: If process $p$ has received $2f+1$ distinct \textit{ready} messages with empty paths from $q$ (i.e. process $q$ has delivered), it can avoid sending related messages to $q$.
%     \item \textbf{MBD.10}: When process $p$ receives a messages with a path $t_0$, and said path is a superpath of a path $t_1$ of an earlier messages, $p$ can ignore the message.
%     \item \textbf{MBD.11}: Use a subset of process of size $\ceil{\frac{N+f+1}{2}} + f$ and $3f+1$ to complete the \textit{echo} and \textit{ready} phase, respectively.
%     \item \textbf{MBD.12}: If a source $s$ has more than $2f+1$ neighbours, it can transmit the \textit{send} message to only $2f+1$ of them instead of all.
% \end{itemize}

\textbf{TODO: talk about other solutions (signatures, trusted nodes, HotStuff BFT) and how they could be applied?
}