\section{System model and problem statement}
\label{system-model}
% Choose one that fits your research best:
% \subsection{Methodology}
% Typically in general research articles, the second section contains a description of the research methodology, explaining what you, the researcher, is doing to answer the research question(s), and why you have chosen this method.
% For purely analytical work this is a description of the data collection or experimental setup on how to test the hypothesis, with a motivation.
% In any case this section includes references to necessary background information.

% \subsection{Formal Problem Description}
% For some types of work in computer science the methodology is standard: analyze the problem (e.g., make assumptions and derive properties), present a new algorithm and its theoretical background, proving its correctness, and evaluate unproven aspects in simulation.
% Then an explanation of the methodology is often omitted, and the setup of the evaluation is part of a later section on the evaluation of the ideas.\footnote{This already shows that there is no single outline to be given for all papers.}
% In this case, explain relevant concepts, theory and models in this section (with references) and relate them to your research question.
% Also this section then typically contains a more precise, formal description of the problem.

% Do not forget to give this section another name, for example after the problem you are solving.

Our model is defined by a set $\Delta=\{p_1, p_2,...,p_N\}$ of N processes, uniquely identified by an identifier $i$ known to all others. In the Byzantine fault-model it is assumed that there are at most $f < \floor{N/3}$ Byzantine nodes, which can exhibit arbitrary behaviour. 

Furthermore, the processes are connected by a network which can be represented by an undirected graph $G=(V,E)$. In this graph every vertex represents a process $p_i$, such that $p_i \in \Delta$, which means $V=\Delta$. It follows that the edges represent the communication links between nodes.
Processes $p_i, p_j \in \Delta$ have a direct communication link if there exists an edge $(v_i, v_j) \in E$, which they can use to directly communicate with each other. If there exists no such link, they will have to rely on other processes to relay their messages. We assume that these links are authenticated, i.e.\jd{,} messages delivered at $p_i \in \Delta$ via edge $(v_i, v_j) \in E$ are guaranteed to originate from $p_j$, and vice-versa. In addition to this, the links are reliable, i.e. messages will always arrive at $p_i$ if and only if $p_j$ sent them over edge $(v_i, v_j)$. However, there is no delivery time or delivery order guarantee, so a link can be synchronous or asynchronous. Graph $G$ is known to all processes, and so are the IDs for every process. Furthermore, it is assumed the network is static, i.e. the network topology does not change, and one or more processes can broadcast messages simultaneously. The processes are used as underlying layer for application code, which receives data from the process when it delivers a message to the application layer.
%, and the network lives for a considerable amount of time.

In order to send message data to others, processes can add information to the message header, which can be used to uniquely identify the message and add protocol specific information.

A Byzantine Reliable Broadcast (BRB) protocol guarantees the following properties:\\
(i) \textbf{Validity}: If process $p_i \in \Delta$ broadcasts message $m$, then every correct process $p_j \in \Delta$ delivers $m$ at some point.\\
(ii) \textbf{No duplication}: A message $m$ broadcast by process $p_i \in \Delta$ is not delivered more than once by every correct process $p_j \in \Delta$.\\
(iii) \textbf{Integrity}: If process $p_j \in \Delta$ delivers message $m$ with sender $p_i$, process $p_i$ has broadcast $m$ in the past.\\
(iv) \textbf{Agreement}: If process $p_i \in \Delta$ delivers message $m$, then $m$ will eventually be delivered by every correct process $p_j \in \Delta$.

% \begin{itemize}
%     \item \textbf{Validity}: If process $p_i \in \Delta$ broadcasts message $m$, then every correct process $p_j \in \Delta$ delivers $m$ at some point.
%     \item \textbf{No duplication}: A message $m$ broadcast by process $p_i \in \Delta$ is not delivered more than once by every correct process $p_j \in \Delta$.
%     \item \textbf{Integrity}: If process $p_j \in \Delta$ delivers message $m$ with sender $p_i$, process $p_i$ has broadcast $m$ in the past.
%     \item \textbf{Agreement}: If process $p_i \in \Delta$ delivers message $m$, then $m$ will eventually be delivered by every correct process $p_j \in \Delta$.
% \end{itemize}

We will be introducing improvements to both Dolev~\cite{dolev}, Bracha~\cite{bracha}, and Bracha-Dolev~\cite{bracha-dolev}, which make different assumptions about the network $G$ and provide different guaranteed properties. 
%\jd{?}

%\jd{why do you repeat the papers' assumptions here? Which assumptions do you make?}

Dolev assumes a network $G$ that is at least $2f+1$-connected.
%, i.e. there are at least $2f+1$ vertex-disjoint paths from every vertex $v_i \in V$ to every vertex $v_j \in (V - \{v_i\})$.
Furthermore, Dolev assumes Reliable Communication (RC) which guarantees the same properties as BRB, except for the \textbf{Agreement} property. Bracha assumes a fully connected network $G$, i.e. for every pair $v_i,v_j \in V$ there exists an edge $(v_i, v_j) \in E$. Unlike Dolev, Bracha guarantees all BRB properties.

We make the same assumptions as the mentioned protocols, while adding topology knowledge and static networks.

\subsection*{Reducing the number of messages}

While all mentioned protocols work well in their designed environments, there is naturally a substantial amount of redundant work.

In the case of Dolev, for example, the network is being flooded with the same message in order to reach all processes over at least $f+1$ vertex-disjoint paths. However, if the network is overprovisioned, i.e. the network is $k$-connected where $k > 2f+1$, the message will take more paths than strictly necessary. In addition to this, processes will send the message to (almost) all neighbours leading to overlapping paths. Even though recent improvements have reduced the amount of required messages 
%from $\mathcal{O}(n!)$ to nearly $\mathcal{O}(n)$~\cite{bonomi2019multihop}
, it is possible this can still be reduced if processes have knowledge of the network topology. The original paper describes a routed version which mitigates this problem. However, the actual creation of the routes is not discussed.

For Bracha there are similar inefficiencies. For example, Bracha uses all nodes to come to an agreement, while only a subset of size $\ceil{\frac{N+f+1}{2}} + f$ is needed in the \textit{echo} phase and a smaller subset of size $3f+1$ for the final \textit{ready} phase.

% The optimizations can be combined and applied to Bracha-Dolev, also reducing the amount of required messages for that protocol.

This paper aims to reduce the number of messages which are transmitted through the network for the three mentioned protocols. This reduces the network usage even further, assuming that all processes know the network topology. In addition to this, it might be possible to improve the delivery complexity of Dolev in the process by taking advantage of the fact that messages will traverse fixed paths. Furthermore, while the general process for handling known topologies for Dolev has been described in the original paper~\cite{dolev}, no actual implementation has been provided, which is something this paper will also do and is an additional contribution of this work.


\section{Background}
\label{background}
In this section we will explain Dolev's and Bracha's protocols, and how they can be combined into Bracha-Dolev.

\subsection*{Dolev}
Dolev's protocol provides reliable communication when the network has authenticated links and is at least $2f+1$-connected.

When a message traverses the network, processes leverage the authenticated links to build a traversal path for each message. Said paths have two purposes: avoiding transmission loops and message verification. 
The former is at play when processes relay messages to their neighbours; a message is forwarded to all neighbours, except to the transmitter and processes which are already present in the path. This is required in order to avoid messages circulating through the network indefinitely.
The paths are also used for verification; a message is delivered whenever it has been received over $f+1$ disjoint paths.

The basis for the correctness for Dolev's protocol lies in Menger's theorem~\cite{menger} which shows that there exist $2f+1$ disjoint paths between every pair of processes if a network is $2f+1$-connected, and the fact that at most $f$ of those paths can contain one or more Byzantine processes. This follows from the fact that a single process can only be in a single path, otherwise said path would not be disjoint, so the worst case scenario is that all $f$ Byzantine processes are spread over all the $2f+1$ disjoint paths, which leaves $f+1$ paths not containing a Byzantine process.

There are multiple ways to verify that a message has traversed $f+1$ disjoint paths, but the problem is often reduced to a max flow problem or a minimum vertex cut problem~\cite{bonomi2019multihop}. \textbf{TODO: refs to set packing and hitting set?} We will elaborate on the first option. In this case paths are modeled as a flow network where every edge has a capacity of one. The maximum flow through the network from source $p_j$ to sink $p_i$ is then equal to the amount of disjoint paths. The max flow can be found with the Ford-Fulkerson algorithm~\cite{ford_fulkerson} for example. 
%\jd{this only works with non Byzantine processes} ---> Why??
Pseudocode for Dolev's protocol for a single message is provided in Algorithm~\ref{background:dolev}.

\begin{algorithm}
  \DontPrintSemicolon
  \SetKwFunction{DInit}{Init}
  \SetKwProg{Fn}{On event}{:}{}
  \Fn{\DInit}{
        delivered = False\;
        paths = $\emptyset$\;
  }
  
  \SetKwFunction{DRecv}{Receive}
  \Fn{\DRecv{$p_{recv}$, $m$, $path$}}{
        $path$ = $path \cup \{p_{recv}\}$\;
        \ForAll{$p_j \in neighbours - path$}{
            transmit($p_j$, $m$, $path$)\;
        }
  
        paths.add($path$)\;

        \uIf{paths contains $f+1$ node-disjoint paths to the origin \textbf{and} delivered = False}{
            deliver($m$)\;
            delivered = True\;
        }
  }
  
  \SetKwFunction{DBrd}{Broadcast}
  \Fn{\DBrd{$m$}}{
        deliver($m$)\;
        delivered = True\;
            
        \ForAll{$p_j \in neighbours$}{
            transmit($p_j$, $m$, $\emptyset$)\;
        }
  }
 \caption{Dolev's Reliable Communication algorithm}
 \label{background:dolev}
\end{algorithm}

\subsection*{Bracha}
Unlike Dolev's protocol, Bracha's protocol requires a fully connected network while guaranteeing all four BRB properties, including the \textbf{Agreement} property. 
%\jd{is it not the case for Dolev? You haven't explained which BRB property is not satisfied by Dolev}
The protocol has three phases: \textit{send}, \textit{echo}, and \textit{ready}.

When a process wants to broadcast a message it sends the payload in a \textit{send} message to all processes, including itself. When a process receives a \textit{send} messages, it responds by transmitting an \textit{echo} message to all processes with the corresponding content. Every process then waits for a minimum of $\ceil{\frac{N+f+1}{2}}$ \textit{echo} messages. 
After this quorum has been reached, or $f+1$ \textit{ready} messages have been received, a process will transmit its own \textit{ready} message to all processes. Finally, a message will be delivered when $2f+1$ corresponding \textit{ready} messages have been received, as can be seen for a single message in Algorithm~\ref{background:bracha}.

\begin{algorithm}[h]
  \DontPrintSemicolon
  \SetKwFunction{BInit}{Init}
  \SetKwProg{Fn}{On event}{:}{}
  \Fn{\BInit}{
        sentEcho = sentReady = delivered = False\;
        echos = readys = $\emptyset$\;
  }
  
  \SetKwFunction{BRecvEcho}{ReceiveEcho}
  \Fn{\BRecvEcho{$p_{recv}$, $m$}}{
        \uIf{\textbf{not} sentEcho}{
            \ForAll{$p_j \in neighbours$}{
                transmit($p_j$, $m$, ECHO)\;
            }
            
            sentEcho = True\;
        }
        
        echos.add($p_{recv}$)\;

        \uIf{len(echos) $\ge$ $\ceil{\frac{N+f+1}{2}}$ \textbf{and not} sentReady}{
            \ForAll{$p_j \in neighbours$}{
                transmit($p_j$, $m$, READY)\;
            }
            
            sentReady = True\;
        }
  }
  
  \SetKwFunction{BRecvReady}{ReceiveReady}
  \Fn{\BRecvReady{$p_{recv}$, $m$}}{
        readys.add($p_{recv}$)\;

        \uIf{len(readys) $\ge f+1$ \textbf{and not} sentReady}{
            \ForAll{$p_j \in neighbours$}{
                transmit($p_j$, $m$, READY)\;
            }
            
            sentReady = True\;
        }
        
        \uIf{len(readys) $\ge 2f+1$ \textbf{and not} delivered}{
            deliver($m$)\;
            delivered = True\;
        }
  }
  
  \SetKwFunction{BBrd}{Broadcast}
  \Fn{\BBrd{$m$}}{
        \ForAll{$p_j \in neighbours$}{
            transmit($p_j$, $m$, SEND)\;
            transmit($p_j$, $m$, ECHO)\;
        }
  }
 \caption{Bracha's authenticated double echo algorithm}
 \label{background:bracha}
\end{algorithm}

\subsection*{Bracha-Dolev}
Dolev's and Bracha's protocol can be combined to achieve BRB guarantees in a multi-hop network, as described in \cite{bracha-dolev}. 

This works by layering the two protocols, where Dolev's protocol forms the bottom layer. This means that every Bracha broadcast is replaced by a Dolev broadcast, and every Bracha receive by a Dolev deliver. This process is shown in figure~\ref{background:bracha-dolev}.

By layering Bracha and Dolev, the latter emulates a fully connected network by enabling the former to reliably reach all processes. However, this means the message complexity of both protocols %, $\mathcal{O}(n^2)$ and $\mathcal{O}(n!)$ respectively,% 
is essentially multiplied.

Instead of simply layering the two protocols, a cross-layer version~\cite{bonomi2021practical} can be used which allows for greater optimization. For comparison, this version can be seen in figure~\ref{background:bracha-dolev}.

\begin{figure}
    \centering
    \begin{subfigure}{.2\textwidth}
      \centering
      \begin{tikzpicture}
        	\begin{pgfonlayer}{nodelayer}
        		\node [style=box] (0) at (-4.25, 6.25) {Application};
        		\node [style=box] (1) at (-4.25, 5.25) {Bracha};
        		\node [style=box] (2) at (-4.25, 4.25) {Dolev};
        		\node [style=box] (3) at (-4.25, 3.25) {Network};
        	\end{pgfonlayer}
        	\begin{pgfonlayer}{edgelayer}
        		\draw [style=directed edge double] (0) to (1);
        		\draw [style=directed edge double] (1) to (2);
        		\draw [style=directed edge double] (2) to (3);
        	\end{pgfonlayer}
        \end{tikzpicture}
    %   \caption{Layering Bracha and Dolev forms the basis for Bracha-Dolev}
    \end{subfigure}
    \begin{subfigure}{.2\textwidth}
      \centering
      \begin{tikzpicture}
        	\begin{pgfonlayer}{nodelayer}
        		\node [style=box] (0) at (-4.25, 6.25) {Application};
        		\node [style=box] (1) at (-4.25, 5.25) {Cross-layer BD};
        		\node [style=box] (2) at (-4.25, 4.25) {Network};
        % 		\node [style=none] (3) at (-4.25, 3.25) {};
        	\end{pgfonlayer}
        	\begin{pgfonlayer}{edgelayer}
        		\draw [style=directed edge double] (0) to (1);
        		\draw [style=directed edge double] (1) to (2);
        	\end{pgfonlayer}
        \end{tikzpicture}
    %   \caption{A cross-layer combination can be used to improve performance~\cite{bonomi2021practical}}
    \end{subfigure}
    \caption{Bracha-Dolev can be implemented by simply layering the two protocols or by using a cross-layer protocol} %~\cite{bonomi2021practical}
    \label{background:bracha-dolev}
\end{figure}