\section{Conclusion}
\label{conclusion}
% Summarize the research question(s) and the answers to the research question(s).
% Make statements.
% Highlight interesting elements.

% Discuss open issues, possible improvements, and new questions that arise from this work; formulate recommendations for further research.

% ideally, this section can stand on its own: it should be readable without having read the earlier sections.
In this paper, we have introduced the Byzantine Reliable Broadcast problem on partially connected networks and fully connected networks where the topology is known to all processes. We started by introducing the current state of the problem and how the original protocols work. We continued by elaborating on how one can find the required paths through the network for Dolev, and then how this information can be used to build routing tables for processes. We then described \textbf{11} modifications to Dolev, Bracha, and Bracha-Dolev, and evaluated each modification separately. When we combine all modifications together, we provide a solution with a lower message complexity and network usage than existing solutions, a reduction of \textbf{XXX}\% and \textbf{XXX}\% respectively with a 12B payload when $N=150$ and $f=20$.
We have concluded that we can indeed reduce the amount of messages transmitted when processes have topology knowledge.

This work can be extended in the future by deploying our modified protocols on real infrastructure to get accurate measurements as opposed to simulations. Furthermore, the disjoint path solver can likely be further optimized by enhancing the path finding and (re)using better suited data structures. It will also be very interesting to modify our protocol so it can function on dynamic networks.
%It might also be interesting to explore the combination of our modifications and (multi-)signatures to further reduce the amount of messages transmitted.
